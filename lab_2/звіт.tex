\documentclass{article}
\usepackage[utf8]{inputenc}
\usepackage[english,ukrainian]{babel}
\usepackage{wrapfig}
\usepackage{graphicx}
\usepackage{type1ec}
\usepackage{mathtext}
\setlength\parindent{24pt}

\graphicspath{ {./images/} }

\PassOptionsToPackage{hyphens}{url}\usepackage{hyperref}
\title{Математичне моделювання та методи оптимізації. Лабораторна 2: Симплекс метод}
\author{Михайло Голуб}
\begin{document}
\maketitle
\newpage

\textbf{Завдання лабораторної роботи:}\\

Використовуючи симплекс метод, знайти максимум/мінімум лінійної функції
при наявності лінійних обмежень за варіантами.\\

\textbf{Варіант 3:}
\begin{equation}
\left\{
\begin{array}{ll}
z \rightarrow max,\\
z = 2x_1 + 3x_2 - x_4,\\
2x_1 - x_2 - 2x_4 - x_5 = 16,\\
3x_1 + 2x_2 + x_3 - x_4 = 18,\\
-x_1+3x_2+4x_4+x_6 = 24,\\
x_j \geq 0, j = \overline{1,6}.
\end{array}
\right.
\end{equation}

\textbf{Хід роботи:}\\

Створено класс Question який містить методи та змінні необхідні для розв'язання вхідної системи:
\begin{itemize}
	\item\texttt{\_\_init\_\_} -- ініціалізатор класу. Приймає на вхід наступні параметри: \texttt{find\_max: bool} (чи це задача максимізації чи мінімізації),  \texttt{main\_func: list} (коефіцієнти головної функції) та \texttt{constrains: list} (матриця з рядками виду \texttt{[$a_{i1}$, $a_{i2}$, ...,} *\texttt{<=} або \texttt{=} або \texttt{>=}*\texttt{, $b_i$]}. Метод, за потреби, перетворює задачу максимізації в задачу мінімізації та усі обмження $\geq$ на $\leq$. Після необхідних перетворень метод записує список коефіцієнтів головної функції та матрицю обмежень в змінні класу.
	\item\texttt{print\_question} -- виводить в консоль систему рівнянь. Приклад виведення завдання варіанту:\\
	\texttt{2*X$_1$ - 3*X$_2$ + 1*X$_4$ --> min\\
 2*X$_1$ - 1*X$_2$ - 2*X$_4$ + 1*X$_5$ = 16\\
 3*X$_1$ + 2*X$_2$ + 1*X$_3$ - 3*X$_4$ = 18\\
 1*X$_1$ + 3*X$_2$ + 4*X$_4$ + 1*X$_6$ = 24}
 	\item\texttt{solve} -- знаходить значення змінних використовуючи \texttt{scipy.optimize.linprog()}, попередньо підготувавши дані для нього. Повертає результат вказаного методу.
 	\item\texttt{print\_solution} -- виводить в консоль знайдені змінні методом \texttt{solve}. Приклад виведення знайдених змінних задачі варіанту:\\
 	\texttt{X$_1$ = 0.5454545454545432\\
X$_2$ = 8.181818181818183\\
X$_3$ = 0.0\\
X$_4$ = 0.0\\
X$_5$ = 23.090909090909093\\
X$_6$ = 0.0}
\end{itemize}

В файлі \texttt{main.py} створено скрипт з трьох команд: створити представник класу \texttt{Question} з інформацією з умови варіанту, \texttt{print\_question()}, \texttt{print\_solution()}.\\

\textbf{Висновок:}\\

Симплекс метод дозволяє розв'язувати задачі мінімізації та максимізації з обмеженнями. Проте, використання вбудованих методів пакетів потребує попередньої підготовки даних.

\end{document}